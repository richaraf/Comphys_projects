\documentclass[norsk,a4paper,12pt]{article}
\usepackage[utf8]{inputenc}
\usepackage{graphicx} %for å inkludere grafikk
\usepackage{verbatim} %for å inkludere filer med tegn LaTeX ikke liker
\usepackage{tabularx}
\usepackage{booktabs}
\usepackage{amsmath}
\usepackage{float}
\usepackage{color}
\usepackage{listings}
\usepackage{physics}
\usepackage{hyperref}
\usepackage{subfig}

\lstset{language=c++}
\lstset{basicstyle=\small}
\lstset{backgroundcolor=\color{white}}
\lstset{frame=single}
\lstset{stringstyle=\ttfamily}
\lstset{keywordstyle=\color{red}\bfseries}
\lstset{commentstyle=\itshape\color{blue}}
\lstset{showspaces=false}
\lstset{showstringspaces=false}
\lstset{showtabs=false}
\lstset{breaklines}
\lstset{postbreak=\raisebox{0ex}[0ex][0ex]{\ensuremath{\color{red}\hookrightarrow\space}}}
\usepackage{titlesec}

\setcounter{secnumdepth}{4}

\titleformat{\paragraph}
{\normalfont\normalsize\bfseries}{\theparagraph}{1em}{}
\titlespacing*{\paragraph}
{0pt}{3.25ex plus 1ex minus .2ex}{1.5ex plus .2ex}


\title{FYS3150 - Computational Physics\\\vspace{2mm} \Large{Project 5}}
\author{\large Richard Andr\'e Fauli\\ Dorthea Gjestvang\\ Even Marius Nordhagen}
\date{December 24, 2016}
\begin{document}

\maketitle
\begin{abstract}
This project aims to simulate so-called quantum dots, electrons trapped in harmonic oscillator (HO) potentials, by using the variational method with Monte Carlo integration. Using two trial wave functions, $\Psi_{T1}$, a pure harmonic oscillator, and $\Psi_{T2}$, which includes a Coulomb repulsion term, we calculate an upper bound ground state energy for different strengths of the HO potential, and the average distance between the two electrons. Furthermore, we test our results against the Virial Theorem.\par

Using $\Psi_{T1}$, we get the exact ground state energy, while $\Psi_{T2}$ gives an upper bound ground state energy that is higher than the exact ground state energy. The average distance between the electrons are higher for $\Psi_{T2}$ compared to $\Psi_{T1}$, and repulsion becomes less important for stronger potentials. The system behaves as expected, and our results correspond to those of Project 2.
\par 

\end{abstract}


\begin{itemize}
\item Github repository containing programs and results are in: \url{https://github.com/richaraf/Comphys_projects/tree/master/Project_5}
\end{itemize}


\section{Introduction}
In this project, we study the theory behind quantum dots, a highly relevant subject in modern physics. A quantum dot is one or two electrons trapped in an quantum well. These are often referred to as arificial atoms, as they, like natural atoms, have discrete energy levels \cite{Nature}.

Our goal in this project is to simulate a quantum dot, modelled as one or two electrons bound in a three-dimensional harmonic oscillator well, both with and without Coulomb interaction. The model is descriped in section \ref{Theory}. \par

Using the Variational Method with Monte Carlo integration, presented in section \ref{Theory} and \ref{Method}, we use two trial wave functions $\Psi_{T1}$ and $\Psi_{T2}$ to calculate an upper bound ground state energy and the mean distance between the two electrons. $\Psi_{T1}$ is a pure harmonic oscillator wave function, while $\Psi_{T2}$ also includes an electron-electron repulsion term. The results are presented in section \ref{Results}. In section \ref{Discussion} we discuss our results, and compare them to the results obtained from Project 2.


\section{Theory} \label{Theory}

\subsection{Variational Method} \label{VariationalMethod}
The variational method is a method in quantum mechanics for estimating an upper limit on the lowest energy state level for a given Hamiltonian \cite{Griffiths}. For any choice of wave function $\Psi$, the following expression holds:

\begin{equation}
    E_0 \leq \frac{\bra{\Psi} H \ket{\Psi}}{\bra{\Psi}\ket{\Psi}} = E_T
    \label{eq:VariationalMethod}
\end{equation}

where $E_T$ is the trial energy. 
Equation \ref{eq:VariationalMethod} states that given a wave function $\Psi$, the calculated expectation value for the energy $E_T$ will always be greater than or equal to the lowest energy, $E_0$. This is trivial to understand: if the chosen $\Psi$ indeed is the ground state $\Psi_0$, then the trial energy is $E_0$. For any other choice of $\psi$, the trial  energy must be higher than the energy of the ground state. \par 
The variational method is particularly useful when the Hamiltonian has complicated eigenfunctions, or no analytical solution at all. A smart choice of a trial wave function $\Psi_T(\alpha)$ where $\alpha$ is the variational parameter, gives the trial energy $E_T(\alpha)$, which is an upper limit on the ground state energy as a function of $\alpha$. The trial energy can thereafter be minimized with respect til $\alpha$, giving an upper limit to $E_0$ that is as low as possible.

\subsection{Quantum Dot Model}
As stated in the introduction, we simulate a quantum dot as two electrons in a three dimensional harmonic oscillator well. We will simulate the electrons both with Coulomb interaction and without. For the non-interacting case, the Hamiltonian $\hat{H}$ is:

\begin{equation}
    \hat{H} = \sum_{i=1}^{2} (-\frac{\hbar^2}{2 m_e}\nabla_i^2 + \frac{1}{2}\hbar \omega^2r_i^2) 
    \label{eq:H_non_interaction_unit}
\end{equation}

where the sum goes over all particles $i$, $\nabla_i^2$ is the Laplace operator for particle $i$, $\omega$ is the harmonic well frequency and $r_i$ is the position vector of particle $i$. 

For the interacting case the electrons repulse each other and we include the Coulumb potential in the Hamiltonian:

\begin{equation}
    \hat{H}_{rep} = \sum_{i<j} \frac{e^2}{4\pi \epsilon_0 r_{ij}}
    \label{eq:H_interaction_unit}
\end{equation}

where $r_{ij}$ is the distance between electron $i$ and $j$, $e$ is the elementary charge, $\epsilon_ 0$ is the permittivity of vacuum, and we sum over $i < j$ to avoid counting interactions twice, as the potential from electron $i$ acting on $j$ is the same as the potential from $j$ acting on $i$. Using natural units, that is scaling such that $\hbar = e = m_e = 1$ and $\epsilon_0 = \frac{1}{4\pi}$, we obtain the scaled Hamiltonians for the non-interacting and the interacting system presented in equation (\ref{eq:H_non_interaction}) and (\ref{eq:H_interaction}):

\begin{equation}
    \hat{H} = \sum_{i=1}^{2} (-\frac{1}{2}\nabla_i^2 + \frac{1}{2} \omega^2r_i^2) 
    \label{eq:H_non_interaction}
\end{equation}

\begin{equation}
    \hat{H}_{rep} = \frac{1}{r_{12}},
    \label{eq:H_interaction}
\end{equation}
where $r_{12}$ is the distance between the two electrons.

We wish to find estimates for the ground state energy using the variational method, presented in subsection \ref{VariationalMethod}. For this, we will use the following trial wave functions:

\begin{equation}
    \Psi_{T1} (r_1, r_2) = C\exp(-\alpha \omega (\frac{r_1^2 + r_2^2}{2}))
    \label{eq:PsiT1}
\end{equation}

\begin{equation}
    \Psi_{T2}(r_1, r_2) = C\exp(-\alpha \omega (\frac{r_1^2 + r_2^2}{2}))\exp(\frac{r_{12}}{2(1+\beta r_{12})})
    \label{eq:PsiT2}
\end{equation}

where C is a normalization constant and $\alpha$ and $\beta$ are a variational parameters. We will use the variational method with $\Psi_{T1}$ on the non-interaction Hamiltonian (\ref{eq:H_non_interaction}), and $\Psi_{T2}$ with the interaction Hamiltonian ((\ref{eq:H_non_interaction}) + (\ref{eq:H_interaction})).

A reader familiar with quantum mechanics may frown upon our choice of trial wave functions, refer to the Pauli principle and say that the wave function of fermions, such as electrons, have to be anti-symmetrical under the interchange of particles. The wave functions presented in equation (\ref{eq:PsiT1}) and (\ref{eq:PsiT2}) are symmetrical, and hence, they cannot represent states of a two-electron system. We can omit this fault by stating that the total wave function $\Phi$ must me anti-symmetrical, and it consists of one spatial part $\Psi$ and one spin part $\chi$:

\begin{equation}
    \Phi = \Psi \cdot \chi 
\end{equation}

As long as the spin part $\chi$ is anti symmetrical, the total wave function $\Phi$ is anti symmetrical, and hence the Pauli principle is satisfied. The only anti symmetrical spin configuration is the spin singlet \cite{Griffiths}, giving that the electrons in our model must have spins pointing in opposite directions. Thus the total spin of the two electrons is 0. Since the spin wave function is known, we choose to only work with the spatial wave function in this project.

\subsection{Energies} \label{Energies}
In order to find the local energy $E_L$ for a given trial function $\Psi_T$ we can use the following expression
\begin{equation}
E_L = \frac{1}{\Psi_T}\hat{H}\Psi_T,
\label{eq:localenergy}
\end{equation}

where $\hat{H}$ is the Hamiltonian. Using equation (\ref{eq:localenergy}) and the first trial wave function (\ref{eq:PsiT1}) we get the local energy to be

\begin{equation}
E_{L1} = \frac{1}{2}\omega^2 (r_1^2 + r_2^2)(1-\alpha^2) + 3\alpha \omega,
\label{eq:EL1}
\end{equation}

which contains both kinetic $T_1$ and potential $V_1$ energy

\begin{equation}
T_1 = 3\alpha \omega -\frac{1}{2}\omega^2\alpha^2\left(r_1^2 + r_2^2\right),
\label{eq:T_1}
\end{equation}

\begin{equation}
V_1 = \frac{1}{2}\omega^2(r_1^2 + r_2^2).
\end{equation}

For the second trial wave function the local energy $E_{L2}$\cite{Project_text} is given as
\begin{equation}
E_{L2} = E_{L1} + \frac{1}{r_{12}} + \frac{1}{2(1+\beta r_{12})^2}\left\{\alpha\omega r_{12}-\frac{1}{2(1+\beta r_{12})^2}-\frac{2}{r_{12}}+\frac{2\beta}{1+\beta r_{12}}\right\},
\label{eq:EL2}
\end{equation}
which we can be expressed as the sum of kinetic energy $T_2$ and potential energy $V_2$ where
$$T_2 = T_1,$$
$$V_2 = E_{L2} - T_2.$$
\subsection{Virial Theorem}
The Virial Theorem provides a general relation between the kinetic and potential energy of a stable system consisting of a finite number of particles over time. For  a pure HO, the virial theorem states that the expectation value of the total kinetic energy $\langle T\rangle$ is equal to the total potential energy $\langle V\rangle$:
\begin{equation}
\langle T\rangle=\langle V\rangle
\label{eq:virial}
\end{equation}
For more complex systems the relation is more complicated, but for an HO we can assume that the relation above holds if the HO potential dominates.


\section{Method} \label{Method}


\subsection{Implementation}
When implementing the code, we made a wave function class which allowed us to specify which wave function we were using, and return the wave function and energy values for specific electron positions.
\par 
\vspace{2mm}

Our trial wave functions are functions of $\vec{r_1}$ and $\vec{r_2}$, and we give our system a random initial state. As both our trial wave functions are proportional to $e^{-\alpha \omega}$, a natural length scale for our system is $\frac{1}{\sqrt{\omega}}$ (which is also our step length as instructed by Morten), as the wave function dies out after $\pm \frac{1}{\sqrt{\omega}}$. We therefore choose a random initial position that is a random number between 0 and 1 multiplied with the length scale for each of the three directions:

\begin{equation}
    position =  random\_number \cdot length\_scale
    \label{random_steps}
\end{equation}

Thereafter, we loop over a number of Monte Carlo Cycles, for each cycle suggesting a random movement for one of the electrons. This random movement follows the same pattern as the generation of the initial position, from equation (\ref{random_steps}). This move is accepted or rejected using the Metropolis algorithm with the probability distribution $|\Psi_T|^2$, presented in subsection \ref{MetropolisAlgorithm}. This way, we can use Monte Carlo Integration for calculating the energy for integration points drawn from our PDF and after $N$ Monte Carlo cycles, we can calculate the trial energy $E_T$.

\subsection{Monte Carlo Integration}
In our project, we will use Monte Carlo Integration. The brute-force way of doing Monte Carlo integration is approximating the integral as a sum, as presented in equation (\ref{eq:MC_int_bruteforce}):

\begin{equation}
    \int_0^1 f(x) dx \approx  \frac{1}{N}\sum_{i=0}^N f(x_i) \quad x \in [0,1]
    \label{eq:MC_int_bruteforce}
\end{equation}

where $f(x)$ is the function we want to integrate and N is the number of integration points. 

A smarter way of doing Monte Carlo integration, is writing the function $f(x)$ as a $g(x)\cdot PDF$ where $PDF$ is a probability distribution, and $g(x)$ is the remaining parts of $f(x)$. Instead of drawing the integration points $x_i$ entirely at random, we can draw them from the PDF, as presented in equation (\ref{MC_int_prob}):

\begin{equation}
    \frac{1}{N}\sum_{i=0}^N g(x_i) PDF \quad x_i \in [-\infty, \infty] \Rightarrow \frac{1}{N} \sum_{i=0}^N g(x_i) \quad x_i \sim PDF
    \label{MC_int_prob}
\end{equation}

This is a more efficient way of doing Monte Carlo integration, as the integration points $x_i$ are chosen from the interval where the function $g(x)$ is interesting.\par 

In our project, we wish to calculate the trial energy, given by equation (\ref{eq:VariationalMethod}). If we rewrite the energy as the local energy $E_L$ from equation (\ref{eq:localenergy}), we can rewrite the inner product as the local energy multiplied with $|\Psi_T|^2$:

\begin{equation}
    \int \Psi_T^* \hat{H} \Psi_T dV = \int E_L |\Psi_T|^2 dV \Rightarrow \frac{1}{N} \sum_{i=0}^N E_L(x_i) \quad x_i \sim |\Psi_T|^2  
\end{equation}

From quantum mechanics, we know $|\Psi_T|^2$ is the probability density of finding a system in a given interval. 

The method of drawing random numbers from a probability distribution is the Metropolis Algorithm, presented in \ref{MetropolisAlgorithm}. 

The implementation is as follows:

\begin{lstlisting}
    E_tot = 0
    
    #N Monte Carlo Cycles
    for(n=0, n < N, n++):
        calculate E_L from function, choose integral points from PDF
        E_tot += E_L
        
    E = E_tot/N
        
    
\end{lstlisting}

\subsection{Metropolis Algorithm} \label{MetropolisAlgorithm}
We wish to draw random numbers from a PDF, and in our program this PDF is $|\Psi_T|^2$. For this, we use the Metropolis algorithm. We know that the electrons are more likely to be found at higher probability density values, i.e larger values of $|\Psi_T|^2$. If the factor $|\Psi_T|^2$ increases when randomly changing one of the electron's position, the algorithm should automatically accept this move. The move should also be accepted if $|\Psi_T|^2$ remains constant.

\par 
\vspace{2mm}

If the probability density decreases, the move is sometimes accepted and sometimes not. Our program calculates the factors $|\Psi_{T_{old}}|^2$ and $|\Psi_{T_{new}}|^2$ for the old and new positions respectively, and their ratio $ m = \frac{|\Psi_{T_{new}}|^2}{|\Psi_{T_{old}}|^2}$ which is less than $1$. Thereafter, the program draws a random number $k$ in the interval $\in [0, 1]$, and compares the ratio $m$ with the random number $k$. If $m \geq k$, the move is accepted. If the $m < k$ the move of the electron is rejected, and the system is reset to the old position. 

\par 
\vspace{3mm}

The implementation would look like the following:

\begin{lstlisting}
    
    Psi_T_old_squared #calculate from function
    
    r_new[i] = r_old[i] + length_scale*random_number
    
    Psi_T_new_squared #calculate from function
    
    if(Psi_T_new_squared >= Psi_T_old_squared):
        accept
    
    else:
        m = Psi_T_new_squared/Psi_T_old_squared
        k = random number in [0,1]
        
        if(m >= k):
            accept
            
        else:
            deny, reset to old positions
    
    
    
    
\end{lstlisting}

\section{Results} \label{Results}
\subsection{Minimal local energy for $\Psi_{T1}$}

We used the trial wave function $\Psi_{T1}$, given by Equation (\ref{eq:PsiT1}), and plotted the energies and the variance in energy as a function of $\alpha$ for different values of $\omega$. The results are shown in Figure (\ref{fig:E_L1_file_omega=1_0}) and (\ref{fig:E_L1_file_omega=0_01}).

\begin{figure} [H]
    \centering
    \includegraphics[scale=0.65]{E_L1_variance_omega=1_0}
    \caption{Local energy and variance plotted  for $\Psi_{T1}$ as a function of $\alpha$, for $\omega = 1.0$.}
    \label{fig:E_L1_file_omega=1_0}
\end{figure} 
We observe that $\alpha = 1.0$ gives the energy $E=3.0$, and the variance is exactly zero.

\begin{figure} [H]
    \centering
    \includegraphics[scale=0.65]{E_L1_variance_omega=0_01_1e8.png}
    \caption{Local energy and variance plotted  for $\Psi_{T1}$ as a function of $\alpha$, for $\omega = 0.01$. We observe that $\alpha = 1.0$ gives the energy $E=3.0$, which is the minimum.}
    \label{fig:E_L1_file_omega=0_01}
\end{figure} 
The variance is close to, but not exactly equal to zero $\alpha$. Similarly, we found that the optimal value of $\alpha$ for $\omega = 0.5$ was $\alpha = 1.0$. We observe that the oscillator is much wider for $\omega=0.01$ than for $\omega=1.00$.

\subsection{Minimal local energy of $\Psi_{T2}$}
For $\omega = 1.0, 0.5, 0.01$ we found the optimal parameters and resulting energy and variance as shown in table \ref{tab:Psi2results}. Contour plots for $\omega = 1.0$ and $\omega =0.01$ is shown in Figure \ref{fig:E_L2_omega=1_0} and \ref{fig:E_L2_omega=0_01} respectively. For $\omega = 0.5$, the contour plot was very similar to Figure \ref{fig:E_L2_omega=1_0}. 
\begin{figure} [H]
    \centering
    \includegraphics[width=12cm]{E_L2_contour_omega=1_0.png}
    \caption{$E_{L2}$ plotted for various $\alpha$- and $\beta$-values when the frequency is set to $\omega=1.0$. The minima was found numerically and is shown as a white dot.}
    \label{fig:E_L2_omega=1_0}
\end{figure}

\begin{table} [H]
\centering
\caption{Optimal parameters $\alpha$ and $\beta$ and resulting energy $E_{L2}$ and variance $\sigma_{E_{L2}}$ for different $\omega$ for $\Psi_{T2}$.}
\begin{tabularx}{\textwidth}{XXXXX} \hline
\label{tab:Psi2results}
$\omega$ & $\alpha$ & $\beta$ & $E_{L2}$ & $\sigma_{E_{L2}}$ \\ \hline
1.0 & 1.0 & 0.25 & 3.7303 & $4.0\cdot 10^{-4}$ \\
0.5 & 1.0 & 0.20 & 2.0001 & $2.0\cdot 10^{-4}$ \\
0.01 & 0.9 & 0.05 & 0.0794 & $4.0\cdot 10^{-6}$ \\ \hline
\end{tabularx}
\end{table}



\begin{figure} [H]
    \centering
    \includegraphics[width=12cm]{E_L2_contour_omega=0_01.png}
    \caption{$E_{L2}$ plotted for various $\alpha$- and $\beta$-values when the frequency is set to $\omega=0.01$. The minima was found numerically and is shown as a white dot.}
    \label{fig:E_L2_omega=0_01}
\end{figure}
When $\alpha=1.00$ and $\beta=0.25$ the local energy is minimal, and holds the value $E_{L2}\approx3.7303$.



\subsection{Average distance between two particles}
We also calculated the average distance between the two electrons for the optimal value of $\alpha$, for different values of $\omega$. The results are presented in Table 

\begin{table} [H]
\centering
\caption{The average distance between the two electrons for $\Psi_{T1}$ and $\psi_{T2}$, for different values of $\omega$.}

\begin{tabularx}{\textwidth}{XXXX} \hline
\label{tab:average_r12}
{\bf } & {\bf $\omega = 1.0$ } & {\bf $ \omega = 0.5 $ } & {\bf $\omega = 0.01$} \\ \hline
{$\Psi_{T1} \quad r_{12}$} & 0.534961 & 0.761391 & 1.70247\\ \hline 
{$\Psi_{T2} \quad r_{12}$} & 0.582162 & 0.834164 & 7.2308\\ \hline 
\end{tabularx}
\end{table}

\subsection{Ratio of kinetic and potential energy}
Finally we have been studying the ratio of kinetic and potential energy for both local energies. We were running the program with $10^7$ Monte Carlo simulation for 100 frequencies in the interval $\omega=[0.01,1.00]$ and got the result presented in Figure (\ref{fig:ratio}).

\begin{figure} [H]
    \centering
    \includegraphics[width=12cm]{energy_ratio_1e7.png}
    \caption{The ratio the kinetic and potential energy for $E_{L1}$ (blue line) and $E_{L2}$ (green line). The program was run for 100 frequencies in the interval [0.01, 1.00] and $10^7$ Monte Carlo simulations.}
    \label{fig:ratio}
\end{figure}
From Figure (\ref{fig:ratio}) we can see that the ratio fluctuates around 1 for $E_{L1}$, and approaches 1 for $E_{L2}$.

We also studied how the ratio evolved for increasingly narrower oscillators (increasing frequencies) for $E_{L2}$, and we observed that the ratio approached 1. For instance the ratio $\langle T\rangle/\langle V\rangle\cong0.92$ when $\omega=100$ and $\langle T\rangle/\langle V\rangle\cong0.98$ when $\omega=1000$.

\subsection{Stability of calculations}
We conducted stability analyses of the calculations for the local energy $E_{L1}$. This was done by finding the average energy for a fixed $\alpha$-value for a various number of Monte Carlo simulations ($N$). In Figure (\ref{fig:stability}) the average energy is plotted for $\alpha=0.95$ with $N$ ranging from $10^{0.5}$ to $10^{8.5}$.
\begin{figure} [H]
    \centering
    \includegraphics[width=12cm]{E_L1_alpha=0_95_stability.png}
    \caption{The average energy for $E_{L2}$ and $\alpha=0.95$ when $N$ ranges from $10^{0.5}$ to $10^{8.5}$.}
    \label{fig:stability}
\end{figure}
The average energy approached a value near 3.004 when $N$ goes to infinity. The calculations are stable from $\sim10^6$. 
\subsection{Comparing results with Project 2}\par 

We compared the values of the ground state energies for $\Psi_{T1}$ and $\Psi_{T2}$ with the ground state energies from Project 2. The result are presented in Table \ref{tab:energies_comparing}: 

\begin{table} [H]
\centering
\caption{The ground state energies from Project 2 compared to the obtained values from Project 5 and the analytical value. Here we have used $\omega = 1.0$}

\begin{tabularx}{\textwidth}{XXXX} \hline
\label{tab:energies_comparing}
{\bf } & {\bf Project 2 } & {\bf Project 5 } & {\bf Analytical} \\ \hline
{$\Psi_{T1} $} & 2.99984 & 3.00000 & 3.00000\\ \hline
{$\Psi_{T2} $} & ? & 3.730 & 3.558 \\ \hline
\end{tabularx}
\end{table}

We did not run the two-electron interacting system for $\omega = 1.0$ in Project 2.

\section{Discussion} \label{Discussion}
Our results from Project 2 showed that for stronger potentials, the Coulomb interaction became negligible \cite{Project_2}. From table \ref{tab:average_r12} we observe that for weaker HO potentials, the average distance $r_{12}$ between the electrons for $\Psi_{T1}$ and $\Psi_{T2}$ are significantly different, while a strong potential gives nearly the same average distance for the two trial wave functions. This is consistent with our results from Project 2, and the explanation is that for weak fields, the electrons are very loosely bound. Because of the Coulomb interaction, they would therefore prefer to be as far away from each other as possible, and they are free to do so. When $\omega$ is large, however, the electrons are forced together.. In the limit when $\omega$ is very large, the average distance $r_{12}$ between the electrons approaches the same value for $\Psi_{T1}$ and $\Psi_{T2}$, because the Coulomb term in $\Psi_{T2}$ becomes negligible.\par 
\vspace{3mm}

For $\Psi_{T1}$ we found the optimal parameter to be $\alpha=1.0$ for all frequencies tested, and we got the exact result which is demonstrated by the variance being $0$. The reason we get the exact result is that the first part of the energy $E_{L1}$ in equation (\ref{eq:EL1}) becomes $0$, which means the energy is independent of the positions of the electrons and is then just $3\omega$. 

We also observed that the distance between the electrons became smaller as the strength of the HO increased, as expected.
\\* \\* \noindent
For the second wave function, the optimal parameters were not constant, but changed depending on $\omega$. From table \ref{tab:Psi2results} we also see that the variance does not become zero as it did for the first wave function. The energies were larger than those from the first wave function, but we see that the ratio of the energies from $\Psi_{T1}$ and $\Psi_{T2}$ moves closer to unity as $\omega$ grows. Which is what one would expect since the repulsion becomes less important when the HO becomes stronger.

Just as for the first wave function we see the distance between the electrons shrink as the well increases in strength, and again we see that the two cases become more alike when $\omega$ increases. \par \vspace{3mm}

When studying the stability of the calculations, we used $E_{L1}$ and assumed that the stability sustained for $E_{L2}$. $\alpha=1.00$ is the only parameter that we have the exact analytical value of, but since the numerical calculations give this value even when $N$ is low, we had to choose a nearby value to study the stability, and we ended up with $\alpha=0.95$. We expected the average energy to approach a value slightly higher than 3.00 when we increased $N$, because the exact value when $\alpha=1.00$ is $E_{L1}=3.00$.\par\vspace{3mm} 

From Table \ref{tab:energies_comparing}, we observe that for $\Psi_{T1}$ we obtain the exact energy value in this project, whereas in Project 2 we got very close. We observe that in Project 2 we got an estimate of the ground state energy that is lower than the actual ground state energy. In Project 2, this was not problematic since we did not use the variational method. The variational method can never give values lower than the ground state energy. \par\vspace{3mm}
For $\Psi_{T2}$ we did not calculate the energy for $\omega =1$ in Project 2. The value we obtain in this project is larger than the analytical energy. This is as expected, because as long as $\Psi_{T2}$ isn't the analytical ground state wave function, the variational method  will give energies that are higher than the analytical ground state energy. 


\section{Conclusion} \label{Conclusion}

\newpage
\section{References}
\begingroup
\renewcommand{\section}[2]{}
\begin{thebibliography}{}
\bibitem{Nature}
  R C Asgoori. 
  Electrons in artificial atoms
  Nature Vol 379 1.February 1996
  \url{http://www.nature.com/nature/journal/v379/n6564/pdf/379413a0.pdf}
\bibitem{Griffiths}
  D J Griffiths (2014)
  Introdction to Quantum Mechanics, Second Edition
\bibitem{Project_2}
  Fauli et. al. (2016)
  Project 2
\bibitem{Project_text}
  Project Description: 					(December 6th 2016)\newline
  \url{https://github.com/CompPhysics/ComputationalPhysics/blob/gh-pages/doc/Projects/2016/Project5/QuantumMonteCarlo/pdf/QuantumMonteCarlo.pdf}
  
  
  

\end{thebibliography}

\end{document}
