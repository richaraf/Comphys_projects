\documentclass[norsk,a4paper,12pt]{article}
\usepackage[utf8]{inputenc}
\usepackage{graphicx} %for å inkludere grafikk
\usepackage{verbatim} %for å inkludere filer med tegn LaTeX ikke liker
\usepackage{tabularx}
\usepackage{booktabs}
\usepackage{amsmath}
\usepackage{float}
\usepackage{color}
\usepackage{listings}
\usepackage{physics}
\usepackage{hyperref}
\lstset{language=c++}
\lstset{basicstyle=\small}
\lstset{backgroundcolor=\color{white}}
\lstset{frame=single}
\lstset{stringstyle=\ttfamily}
\lstset{keywordstyle=\color{red}\bfseries}
\lstset{commentstyle=\itshape\color{blue}}
\lstset{showspaces=false}
\lstset{showstringspaces=false}
\lstset{showtabs=false}
\lstset{breaklines}
\lstset{postbreak=\raisebox{0ex}[0ex][0ex]{\ensuremath{\color{red}\hookrightarrow\space}}}
\usepackage{titlesec}

\setcounter{secnumdepth}{4}

\titleformat{\paragraph}
{\normalfont\normalsize\bfseries}{\theparagraph}{1em}{}
\titlespacing*{\paragraph}
{0pt}{3.25ex plus 1ex minus .2ex}{1.5ex plus .2ex}


\title{FYS3150 - Computational Physics\\\vspace{2mm} \Large{Project 5}}
\author{\large Richard Andr\'e Fauli\\ Dorthea Gjestvang\\ Even Marius Nordhagen}
\date{December 24, 2016}
\begin{document}

\maketitle
\begin{abstract}
This project aims to simulate so-called quantum dots, electrons trapped in harmonic oscillator potentials, by
\end{abstract}


\begin{itemize}
\item Github repository containing programs and results are in: \url{https://github.com/richaraf/Comphys_projects/tree/master/Project_5}
\end{itemize}


\section{Introduction}
In this project, we study the theory behind quantum dots, a highly relevant subject in modern physics. A quantum dot is one or two electrons trapped in an quantum well. These are often referred to as arificial atoms, as they, like natural atoms, have discrete energy levels \cite{Nature}.

Our goal in this project is to simulate a quantum dot, modelled as one or two electrons bound in a three-dimensional harmonic oscillator well.


\section{Theory}

\subsection{Variational Method}
The variational method is a method in quantum mechanics for estimating an upper limit on the lowest energy state level for a given Hamiltonian \cite{Griffiths}. For any choice of wave function $\psi$, the following expression holds:

\begin{equation}
    E_0 \leq \frac{\bra{\psi} H \ket{\psi}}{\bra{\psi}\ket{\psi}}
    \label{eq:VariationalMethod}
\end{equation}

Equation \ref{eq:VariationalMethod} states that given a wave function $\psi$, the calculated expectation value for the energy will always be greater than or equal to the lowest energy, $E_0$. This is trivial to understand: if the chosen $\psi$ indeed is the ground state $\psi_0$, then the energy is $E_0$. For any other choice of $\psi$, the energy must be higher than the energy of the ground state. \par 
The variational method is particularly useful when the Hamiltonian has complicated eigenfunctions, or no analytical solution at all. A smart choice of a trial wave function $\psi_T(\alpha)$ where $\alpha$ is the variational parameter, gives the trial energy $E_T(\alpha)$, which is an upper limit on the ground state energy as a function of $\alpha$. The trial energy can thereafter be minimized with respect til $\alpha$, giving an upper limit to $E_0$ that is as low as possible.

\subsection{Quantum Dot Model}
As stated in the introduction, we simulate a quantum dot as two electrons in a three dimensional harmonic oscillator well. We will both simulate the electrons when neglecting Coulomb interaction between the two electrons, and when including this repulsion. For the non-interacting case, the Hamiltonian $\hat{H}$ is:

\begin{equation}
    \hat{H} = \sum_{i=1}^{2} (-\frac{\hbar^2}{2 m_e}\nabla_i^2 + \frac{1}{2}\hbar \omega^2r_i^2) 
    \label{eq:H_non_interaction_unit}
\end{equation}

where the sum goes over all particles $i$, $\nabla_i^2$ is the Laplace operator for particle $i$, $\omega$ is the harmonic well frequency and $r_i$ is the position vector of particle $i$. 

For the interacting case, the electrons notices a repulsion from the Coulomb interaction, and we have to include an additional term in the Hamiltonian:

\begin{equation}
    \hat{H_{rep}} = \sum_{i<j} \frac{e^2}{4\pi \epsilon_0 r_{ij}}
    \label{eq:H_interaction_unit}
\end{equation}

where $r_{ij}$ is the distance between electron $i$ and $j$, $e$ is the elementary charge, $\epsilon_ 0$ is the permittivity of vacuum, and we sum over $i < j$ to avoid counting interactions twice, as the potential from electron $i$ acting on $j$ is the same as the potential from $j$ acting on $i$. Using natural units, that is scaling such that $\hbar = e = m_e = 1$ and $\epsilon_0 = \frac{1}{4\pi}$, we obtain the scaled Hamiltonians for the non-interacting and the interacting system presented in equation \ref{eq:H_non_interaction} and \ref{eq:H_interaction}:

\begin{equation}
    \hat{H} = \sum_{i=1}^{2} (-\frac{\hbar^2}{2 m_e}\nabla_i^2 + \frac{1}{2}\hbar \omega^2r_i^2) 
    \label{eq:H_non_interaction}
\end{equation}

\begin{equation}
    \hat{H_{rep}} = \sum_{i<j} \frac{e^2}{4\pi \epsilon_0 r_{ij}}
    \label{eq:H_interaction}
\end{equation}

\subsection{Energies} \label{Energies}
In order to find the local energy $E_L$ for a given trial function $\Psi_T$ we can use the following expression
\begin{equation}
E_L = \frac{1}{\Psi_T}\hat{H}\Psi_T,
\label{eq:localenergy}
\end{equation}
where $\hat{H}$ is the Hamiltonian. Using equation (\ref{eq:localenergy}) and the first trial wave function (\ref{eq:PsiT1}) we get the local energy to be 
\begin{equation}
E_{L1} = \frac{1}{2}\omega^2 (r_1^2 + r_2^2)(1-\alpha^2) + 3\alpha \omega,
\label{eq:EL1}
\end{equation}
which contains both kinetic $T_1$ and potential $V_1$ energy
\begin{equation}
T_1 = 3\alpha \omega -\frac{1}{2}\omega^2\alpha^2\left(r_1^2 + r_2^2\right),
\label{eq:T_1}
\end{equation}
\begin{equation}
V_1 = \frac{1}{2}\omega^2(r_1^2 + r_2^2).
\end{equation}



\section{Method}
\section{Discussion}
\section{Conclusion}

\newpage
\section{References}
\begingroup
\renewcommand{\section}[2]{}
\begin{thebibliography}{}
\bibitem{Nature}
  R C Asgoori. 
  Electrons in artificial atoms
  Nature Vol 379 1.February 1996
  \url{http://www.nature.com/nature/journal/v379/n6564/pdf/379413a0.pdf}
\bibitem{Griffiths}
  D J Griffiths (2014)
  Introdction to Quantum Mechanics, Second Edition
  
  

\end{thebibliography}

\end{document}
