\documentclass[norsk,a4paper,12pt]{article}
\usepackage[utf8]{inputenc}
\usepackage{graphicx} %for å inkludere grafikk
\usepackage{verbatim} %for å inkludere filer med tegn LaTeX ikke liker
\usepackage{amsmath}
\usepackage{float}
\usepackage{color}
\usepackage{listings}
\usepackage{hyperref}
\lstset{language=c++}
\lstset{basicstyle=\small}
\lstset{backgroundcolor=\color{white}}
\lstset{frame=single}
\lstset{stringstyle=\ttfamily}
\lstset{keywordstyle=\color{red}\bfseries}
\lstset{commentstyle=\itshape\color{blue}}
\lstset{showspaces=false}
\lstset{showstringspaces=false}
\lstset{showtabs=false}
\lstset{breaklines}
\lstset{postbreak=\raisebox{0ex}[0ex][0ex]{\ensuremath{\color{red}\hookrightarrow\space}}}


\title{FYS-3150: Project 2}
\author{Richard Fauli\\ Dorthea Gjestvang\\ Even Nordhagen}
\date{\today}
\begin{document}

\maketitle

\begin{abstract}
\end{abstract}
\begin{itemize}
\item Github repository containing programs and results are in: \url{https://github.com/richaraf/Comphys_projects/tree/master/Project_2}
\end{itemize}
\section{Introduction}
The purpose of this project is.. 

\section{Theory}
<<<<<<< HEAD

If we consider a set of vectors $\{\textbf{v}_i\}$, an orthogonal matrix U and the transformation $\textbf{w}_i = U\textbf{v}_i$. The dot product for our set of vectors an be expressed as $$\textbf{v}_j^T\textbf{v}_i,$$ which in the case of $\{\textbf{v}_i\}$ being an orthonormal basis means that $\textbf{v}_j^T\textbf{v}_i = \delta_{ij},$ where $\delta_{ij}$ is the Kronecker delta. The dot product of an orthogonal transformation is preserved
$$\textbf{w}_j^T \textbf{w}_i = (U\textbf{v}_j)^TU\textbf{v}_i = \textbf{v}_j^TU^TU\textbf{v}_i = \textbf{v}_j^T\textbf{v}_i,$$ which also means that orthogonality is preserved in the case where $\{\textbf{v}_i\}$ is an orthogonal basis.
=======
If we consider a set of vectors $\textbf{v}_i$ and an orthogonal matrix U. 

>>>>>>> 5f18e82850d0f3b9ff84cabf4bf4ff8e47cd4cbf
\section{Method}
In this project we are using Jacobi's method, which transforms a matrix A into a diagonal matrix D with the eigenvalues of A on the diagonal. This is done by repeat setting the largest non-diagonal element to zero until all the non-diagonal elements are smaller than a tolerance $\varepsilon$. From linear algebra we know that a symmetric matrix always can be transformed into a diagonal matrix, so this should be possible for our specific matrix. For doing this we need a matrix S which sets the largest non-diagonal element to zero:
\begin{equation}
S^T A S=B
\end{equation}
Where B is the matrix 
\section{Results}
\section{Discussion}
\section{References}
\begingroup
\renewcommand{\section}[2]{}
\begin{thebibliography}{}
\bibitem{MHJ15}
  Morten Hjorth-Jensen.
  Computational Physics, Lecture Notes Fall 2015.
  Department of Physics, University of Oslo.
  August 2015.

\end{thebibliography}
\endgroup
\section{Code attachment}
\end{document}