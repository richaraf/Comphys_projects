\documentclass[norsk,a4paper,12pt]{article}
\usepackage[utf8]{inputenc}
\usepackage{graphicx} %for å inkludere grafikk
\usepackage{verbatim} %for å inkludere filer med tegn LaTeX ikke liker
\usepackage{amsmath}
\usepackage{float}
\usepackage{color}
\usepackage{listings}
\usepackage{hyperref}
\lstset{language=c++}
\lstset{basicstyle=\small}
\lstset{backgroundcolor=\color{white}}
\lstset{frame=single}
\lstset{stringstyle=\ttfamily}
\lstset{keywordstyle=\color{red}\bfseries}
\lstset{commentstyle=\itshape\color{blue}}
\lstset{showspaces=false}
\lstset{showstringspaces=false}
\lstset{showtabs=false}
\lstset{breaklines}
\lstset{postbreak=\raisebox{0ex}[0ex][0ex]{\ensuremath{\color{red}\hookrightarrow\space}}}


\title{FYS3150 - Computational Physics\\\vspace{2mm} \Large{Project 3}}
\author{\large Richard Andr\'e Fauli\\ Dorthea Gjestvang\\ Even Marius Nordhagen}
\date{\today}
\begin{document}

\maketitle

\begin{abstract}
\end{abstract}
\begin{itemize}
\item Github repository containing programs and results are in: \url{https://github.com/richaraf/Comphys_projects/tree/master/Project_3}
\end{itemize}
\section{Introduction}
\section{Theory}
\section{Method}
\subsection{Ordinary differential equation solvers}
In this project we use two different ODE solvers, the Euler-Cromer method and the velocity Verlet method. Both are derived using Taylor expansions to respectfully second and third order. 
\subsubsection{Euler-Cromer}
\subsubsection{Velocity Verlet}
We are looking at a system with position $x_i$ and velocity $v_i$ at step number $i$. By using third order Taylor expansions we get that the position at step number $i+1$ becomes \begin{equation}
x_{i+1} = x_i + h v_i + \frac{h^2}{2}v_i^{(1)} + {\cal O}(h^3),
\label{eq:velVerx}
\end{equation}
where we have used that $x_i^{(1)} = v_i$ and $h$ is the step length. Similarly we can express the velocity as
\begin{equation}
v_{i+1} = v_i + h v_i^{(1)} + \frac{h^2}{2}v_i^{(2)} + {\cal O}(h^3).
\label{eq:vel2ndorder}
\end{equation}
Since the first derivative of the velocity, acceleration is often known analytically we can use a second order Taylor expansion to find an estimate for the acceleration at time step $i+1$
$$v_{i+1}^{(1)} = v_i^{(1)} + hv_i^{(2)} + {\cal O}(h^2), $$
which can for small $h$ be written as 
\begin{equation}
hv_i^{(2)} \approx v_{i+1}^{(1)}- v_i^{(1)}.
\label{eq:hv1approx}
\end{equation}
We can then insert equation (\ref{eq:hv1approx}) into equation (\ref{eq:vel2ndorder}) to get the approximation which we will use for velocity
\begin{equation}
v_{i+1} = v_i + \frac{h}{2}\left(v_{i+1}^{(1)} + v_i^{(1)}\right) + {\cal O}(h^3),
\label{eq:velVerv}
\end{equation}
alongside equation (\ref{eq:velVerx}) for the position. 

When we have an analytical expression for the force, we can use Newton's second law of motion (\ref{n2l}) to find accelerations $v_{i+1}^{(1)}$ and $v_{i}^{(1)}$. Numerically this can be implemented by:
\begin{lstlisting}
for(i, i++):
	a(i) = computeForce/mass
	v(i+1/2) = v(i) + h/2 * a(i)
	x(i+1) = x(i) + h * v(i+1/2)
	// calculate force at the new time step i+1
	a(i+1) = computeForce/mass
	v(i+1) = v(i+1/2) + h/2 * a(i+1)
\end{lstlisting}

\section{Results}
\section{Discussion}
\section{Conclusion}
\section{References}
\begingroup
\renewcommand{\section}[2]{}
\begin{thebibliography}{}
\bibitem{MHJ15}
  Morten Hjorth-Jensen.
  Computational Physics, Lecture Notes Fall 2015.
  Department of Physics, University of Oslo.
  August 2015.
\end{thebibliography}
\endgroup
\end{document}
